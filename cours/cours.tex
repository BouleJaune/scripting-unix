\documentclass[11pt,a4paper]{article}
\usepackage[utf8x]{inputenc}
\usepackage[T1]{fontenc}
%\usepackage{gentium}
\usepackage{mathptmx} % Use Times Font


\usepackage[pdftex]{graphicx} % Required for including pictures
\usepackage[french]{babel} 
\usepackage[pdftex,linkcolor=black,pdfborder={0 0 0}]{hyperref} % Format links for pdf
\usepackage{calc} % To reset the counter in the document after title page
\usepackage{enumitem} % Includes lists

\frenchspacing % No double spacing between sentences
\linespread{1.2} % Set linespace
\usepackage[a4paper, lmargin=0.1666\paperwidth, rmargin=0.1666\paperwidth, tmargin=0.1111\paperheight, bmargin=0.1111\paperheight]{geometry} %margins
%\usepackage{parskip}

\usepackage[all]{nowidow} % Tries to remove widows
\usepackage[protrusion=true,expansion=true]{microtype} % Improves typography, load after fontpackage is selected

\usepackage{lipsum} % Used for inserting dummy 'Lorem ipsum' text into the template


%-----------------------
% Set pdf information and add title, fill in the fields
%-----------------------
\hypersetup{ 	
pdfsubject = {},
pdftitle = {},
pdfauthor = {}
}

%-----------------------
% Begin document
%-----------------------
\begin{document} %All text i dokumentet hamnar mellan dessa taggar, allt ovanför är formatering av dokumentet
\tableofcontents
\section{Introduction}

Les objectifs de la formation\\ 

Connaître les caractéristiques des principaux outils de scripting Unix/Linux

Savoir lire des scripts Unix/Linux écrits en Shell, Perl, Python Ruby ou AWK

Être capable d'écrire des scripts simples d'exploitation Unix/Linux

Comprendre comment choisir l'outil le plus adapté pour résoudre un problème particulier
\\

Prez perso et demander le niveau des gens

\section{Le Shell - les bases}

\subsection{Qu'est ce qu'un shell}

Un shell Unix est une interface homme machine (IHM) en ligne de commande (CLI). Il fournit à la fois un langage de commandes interactives et un langage de scripting. Le shell traite des commandes ou scripts.\\ 
Il ne faut pas confondre un shell avec un terminal. Un terminal était initialement physiquement un écran et un clavier. Aujourd'hui lorsque l'on parle de terminal on parle d'émulateur de terminal, c'est une catégorie de logiciels permettant de fournir un GUI pour lancer \textit{des} shells (bash, python, zsh, fish, powershell, ruby ...). \\ 
\'Emulateur de terminaux connus : alacritty, Windows Terminal, urxvt, GNOME Terminal, PuTTY.

La confusion est courante car sur Windows historiquement le nom du shell et de l'émulateur de terminal étaient les mêmes (cmd, powershell...), ce n'est plus le cas avec Windows 11 et le Windows Terminal.\\


\subsection{Différents shells Unix}
Principaux shells, principales différences, tout pareil osef, focus sur Bash.
Abus de langage bash=shell unix

\subsection{Rappels Linux essentiels pour le scripting}
Supposes qu'ils ont un minimum les bases, donc juste l'essentiel pour scrpiting
stdin / stdout
return code, 
droits fichiers, +x, shebang

\subsection{Intérets des Shells Unix}
Le shell malgré sa syntaxe archaique et ses fonctionnalités limitées vis à vis de langages interprétés à usage général a tout de même encore des avantages.

\subsubsection{Actions uniques}
Utile lorsque l'on veut faire une tâche relativement simple que quelques fois. 
Exemple: extraire des données textes et les process sommairement
cat / sed / cut / tr ...

\subsubsection{Manipulation de binaires}
L'une des force de Bash par rapport à d'autres langages et sa facilité à manipuler directement des binaires. En python, par exemple, on peut aussi manipuler des binaires, mais on sent clairement que cela est moins pensé pour.
Exemple : Utiliser un exécutable propriétaire (import baroc), utiliser l'output d'un script python et l'injecter ailleurs etc


\subsubsection{Manipulation de texte}
Tout est texte
Exemple : lire des logs

\subsubsection{Manipulation OS linux}
les deux derniers ~= Manipulation linux pur car linux déjà tout fichiers et texte + binaires unix


Philo kiss, pleins de petits binaires linux avec une seule tache, trop complexe go python

\subsection{"flow" de texte}
Pipes !!

\subsection{Détails syntaxiques}
\subsubsection{Redirections et globs (*)}
\subsubsection{Structures de contrôle}
if then else fi
while do done
until do done
for do done
case esac

test et [ et [[

\subsubsection{Variables}


\subsubsection{Fonctions}
\subsubsection{Arguments parsing}
- vs --

\subsubsection{Binaires utiles}
Force et efficience des binaires linux
cut, cat, echo, grep, tr, sed, xargs, tail, df, ls
diff

man ! RTFM
\subsection{Exercices}
Récupérer la liste des pourcentages de remplissages des filesystems

\subsubsection{Script}

\subsubsection{Sans scripts}

df | tail -n +2 | tr -s " " | cut -d " " -f 5

Puissance de juste one commande + pipe + redirections

Le Shell POSIX/ISO\\
L'écriture de script Shell \\
Activation des commandes POSIX/ISO\\
Les caractères spéciaux (jockers, échappements, redirection)\\
Les variables\\
Les structures de contrôle\\

\section{Le langage Perl - les bases}
Prez, utilité de nos jours, spécifité

Présentation de Perl\\
Les variables scalaires, les tableaux, les opérateurs\\
Les instructions de contrôle\\
Les tableaux associatifs (hash)\\

\section{Le langage Ruby - les bases}
Présentation de Ruby\\
Les variables\\
Les chaînes de caractères\\
Les structures de contrôle\\
Les tableaux, les itérateurs - Les hash\\

\section{Le langage Python - les bases}
Prez, importance de python, pourquoi (communauté), scope (ansible, blabla, 3600 trucs)

Présentation de Python\\
Variables et expressions\\
Les tableaux, les chaînes de caractères\\
Les instructions de contrôle\\
Les dictionnaires (hash)\\

\section{Les expressions régulières (RegExp)}
regex car tout est texte
Importance de grep et sed 
RegExp en Shell (via grep et sed)\\
RegExp en Perl (normes)\\
RegExp en Python (module re mais osef)\\

\section{La modularité en Shell, Perl, Python et Ruby}

parler de direnv, de requirements.txt, de venv python, de classes python (osef un peu)
y a des trucs pour shell (bpkg) mais pas le but, illogique, shell = spécifique task

Les fonctions\\
Les paquetages\\
L'approche objet\\
Utilisation de bibliothèques externes\\



\section{La programmation parallèle en Shell, Perl, Python et Ruby}

différentes concurrency
xargs pipes et parallel pour shell
multithreading et async pour python

\section{Résoudre des problèmes avec le Shell, Perl, Python et Ruby}

Ecrire des scripts d'exploitation (activer une application, les signaux, ...)\\
Manipuler des fichiers\\
Faire des calculs\\
Ecrire des CGI Web\\
Accéder à des bases de données\\
Manipuler des fichiers XML (parsing, validation, création)\\
Créer des applications réseaux TCP/IP\\

Ptit serveur web
Génération de fichier dans un dossier
service shell filewatcher qui trigger un truc
xargs et des pipes pour multiprocessing
perl ou grep pour regex
awk pour un truc

9 - AWK : un sous-ensemble POSIX/ISO du langage Perl

10 - Conclusion

Quel outil pour quoi faire?



\subsection*{Subtitle}
\end{document}
